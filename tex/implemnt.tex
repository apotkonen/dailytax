%------------------------------------------------------------------------------
%
%	implemnt.tex Document implemenation part
%
%	INCLUDE FILE FOR LaTeX2e DOCUMENT
%
%	AUTHOR: Ari Potkonen /JARVENPAA/ Mon Jun 28 2022
%------------------------------------------------------------------------------
%         1         2         3         4         5         6         7
%123456789012345678901234567890123456789012345678901234567890123456789012345678
%-BEGIN OF INCLUDE FILE--------------------------------------------------------
\chapter{Implementation}
\label{implementation}
\index{implementation}

Implementation depends on from solution purpose.
For demonstration simple solution without any availability
and security requirements is enough as long basic operations can be done.
National solution level there are a lot of development needs from model itself
and from possible other integrations not seen here.
Implementation is most likely embedded into existing systems besides operation ones.
So it would be possible to run new process besides old with correct current data.
Hopefully this more automatic,
less bureaucratic system implementation is started this time
\cite{BasicIncomeInit}\-\cite{LiberaPerustili}\-\cite{A2PalkkaerotIlta}\-\cite{UniversalCredit}\-\cite{UniversalBasicIncome}.
Global solution needs serious planning from process and possible other tax areas
and toll processes maybe wanted to bring together into new
"global rolling tax day" -perspective created solution family,
capable to handle value added tax, even scaled to astronomical level \cite{LTEonMoon}.
Maybe too premature idea here,
but still good keep in mind while planning different stages implementation
so that don't set up any showstoppers for further development on celestial scale,
including at least near orbit's, Moon and Mars.

\section{Initial situation}
\label{db_initial_situation}
\index{database, initial situation}
This is simplified model from existing income tax,
where there are three register keepers,
municipalities and state government having taxing rights.
Province and union taxes are embedded into state tax.
Taxation happens semi-automatically at once of year.
System is rigid, inflexible and usually cause uncertainty to peoples economic situation,
specially at autumn near year-end.
Every corrective adjustment needs extra activity,
which is exhaustive if you are already in tight unexpected financial position due reason or other.
Figure \ref{fig:DbInitial} on page \pageref{fig:DbInitial}.
Tax administration has some idea from needed change \cite{SemiAutomaticWitholdingRate},
but as long as it is based to fixed tax period year,
it still has tax period edge effect we should avoid.
Needed laws changes take long and therefore we has to have much more proactive
technososioeconomic optimized vision from future.

%keepaspectratio=true ?
\begin{figure}
 \begin{center}
  \includegraphics[height=\textwidth,width=\textheight,angle=90]{figures/dtpginit.eps}
  \caption{Initial situation with yearly tax}
  \label{fig:DbInitial} \index{database, initial situation}
 \end{center}
\end{figure}

\section{Theory part speculations}
\label{db_theory_speculations}
\index{database, theory testing}
There are old income tax tables fed into system, province tax is separated from state tax,
and created possibility to separate union taxes. Community tax, besides municipal taxes,
as they are under FinLex. This configuration makes possible to play,
compare old and new solutions.
Still lacking statistics or proper distribution form,
which can be used to determine balance point for given social support.
Anyhow, usable to play with when initialized, fed with the law given values.
Figure \ref{fig:DbSpeculation} on page \pageref{fig:DbSpeculation}. 

\begin{figure}
 \begin{center}
  \includegraphics[width=\textwidth,height=\textheight]{figures/dtpgdoc.eps}
  \caption{Theory part speculations between old and new}
  \label{fig:DbSpeculation} \index{database, mixed mode}
 \end{center}
\end{figure}

\section{Demonstration}
\label{implementation_demo}
\index{implementation demo}
Target state is presented on figure \ref{fig:DbDemoTarget} at page \pageref{fig:DbDemoTarget}.
It will be able to handle income tax and other taxes too.
Resolution time is below two days globally.
Mostly clearing can be done during 24 hours at latest during next 48 hours for global transactions.

TBD (To Be Done) maybe.

\begin{figure}
 \begin{center}
  \includegraphics[height=\textwidth,width=\textheight,angle=90]{figures/dtpgtarg.eps}
  \caption{Demo DB Target situation for tax calculations}
  \label{fig:DbDemoTarget} \index{database, demo target}
 \end{center}
\end{figure}

\section{National version}
\label{implementation_national}
\index{implementation national}
Someone has to do this! There is automatic withholding percentage correction
proposal for the yearly tax, which is better than nothing, but not enough.
Tax legislation and tax system has to be changed, improved.

\section{Global version}
\label{implementation_global}
\index{implementation global}
Maybe some day if financed?

\section{Beyond taxation}
\label{beyond_taxation}
\index{beyond taxation}
This document has been concentrated to taxation and basic social support
which can be integrated, automated with taxation.
Besides this proposed highly automated basic social support with daily tax,
there are still lot need for discretionary support mostly due to ilness,
disability etc. special reasons. In order to understand the problem field,
we have to look a little at the model of the current setup
and fastly changing situation we live in.

\subsection{Existing architecture}
\label{existing_architecture}
\index{existing architecture}

In Finland the Digital and Population Data Services Agency (DVV)
is kind of responsible from population data details
and digitalization services generally, and they domain is "dvv.fi".
Properties are registered under National Land Survey (NLS) services.
Income register and taxation details are on tax domain "vero.fi".
Social welfare and healthcare sector domain is "kanta.fi"
holding patient data repository, diagnosis, prescriptions etc.
Then there is the Social Insurance Institution of Finland (KE\-LA),
"kela.fi" domain paying in this document with the taxation automation proposed support
and many other discretionary supports. When looking KE\-LA from architecture documents \cite{SYPLJULK},
you clearly see that it's describing current situation.
From Figure 3.1 Social and health information managment central players \cite{SYPLJULK_Kuva_3.1}
you can notice that KE\-LA's explicit role as social and health insurance company from customer financing perspective is left out, forgotten.
Kela's roles are only mentioned on describing text.

\begin{figure}
 \begin{center}
  \epsfig{figure=figures/players.eps,width=\textwidth}
  \caption{Players}
  \label{fig:players} \index{players}
 \end{center}
\end{figure}

This significantly affects to digitalized customer process planning,
because so central player is only iplicitly visible,
when thinking KE\-LA's role as sosial and health financial services provider.
Every person doing digitalized process planning has to notice this
when dealing with KE\-LA's roles and one is left out,
during new digitalized service processes creation.

\subsection{Initial situation}
\label{initial_situation}
\index{initial situation}

Initial situation,
today's baseline is where processes are mostly just digitized version from old pen,
paper and physicall access versions from history behind thirty years.
There are lot of processes which are not digitalized at all,
meaning in practice overall technososioeconomical optimization,
fully rewised, rewritten, simplified processes.

\subsection{Practical examples}
\label{practical_examples}
\index{practical examples}

\subsubsection{Elderly peoples example}
When elderly people is under counties wellbeing services
and assessment of the need is done by for this purpose named nurce or doctor.
If home service need is detected,
this detection has to be saved to medical records base "KANTA"
and to be marked as possible cost/support affecting decision.
Decision which could directly be used to same organization "Kela",
the Social Insurance Institution of Finland managed costs reimburcements reasoning,
at least with patien customer, this elder people,
promise to allow sensitive information cross organization use
like financing part from wellbeing services,
pointed service provider services used.
Reality is from last millenium for pen, paper and physical access citizens.
They ask relatives, perhaps from different side of country to help with these things.
Relatives first need computer, network connection, working printer to print papers,
and then someone possible travels severel hundred kilometers carry laptop with,
purchase printer from market and print those papers, fill those with elder people and post to Kela,
which basically could then look decision details from Kanta, but no,
they reply with mail that elder people has to deliver doctors statement during next three weeks or request is cancelled.
This paper then comes to elder people home,
and when relatives come again in place during one month this three weeks is just passed and request is cancelled.
In practice elder people has no access to medical records
and not own paper copy either because demand to arrange service is not given as paper for elder people even requested to arrange support.
So in practice relatives has to request time for doctor again and arrange again new trip over several hundred kilomenters to carry elder people to doctor to dig out decision from Kanta or do new one now with certificate on paper for Kela.
See figure \ref{fig:burdofpr} on page \pageref{fig:burdofpr}.
Yep, Kela does very effectively these decicions to neglect reasoned support,
but is anyone checked process from elderly persons user experience perspective --
so from Pen, Paper, Physical-access Peoples Perspective?
This is severe wasting of overall resources available\cite{ElakelaisetRynJanKoskimiesJaElakeliitonIreneVuorisalo}\-\cite{ElakelaisetEivatTieda}.
Kela has to have capability to check medical records from Kanta with the permisson of elder people given at initial request without asking older people and in practice hes relatives to arrange second evaluation for need what counties wellbeing services are already done!
This is digized manual process,
causing more load and delay than original process before computers utilized at all.
Because all is done twice, first digitally and then manually, even decisions are done twice,
first time to initiate service, and then again to get reasoned reimburcement.
It doesn't help anything to use artificial intelligence\cite{PerukirjaTukiehdotuksetTekoaly}
to replace human application decicion making, because the whole process is crippled,
and currect artificial intelligence doesn't yet have rights nor intelligence to replace whole process yet.
Therefore this process has to be digitalized so that done decisions are on Kanta
and Kela's human (or artificial) intelligence is able to look decisions reasoning directly from there,
at least with information owner given permission given in related suomi.fi-application from!
Now form doesn't even include question from permission to look Kanta to see
well being area records for decision\cite{LifeEventDetailDrivenProcessQuides}.

\begin{figure}
 \begin{center}
  %\includegraphics[width=\linewidth]{figures/burdofpr.fig}
  \epsfig{figure=figures/burdofpr.eps,width=\textwidth}
  \caption{Burden of proof}
  \label{fig:burdofpr}
  \index{process digitized burden of proof}
 \end{center}
\end{figure}

See "Digitalized"-process figure \ref{fig:digitalized} on page
\pageref{fig:digitalized}, how much less it uses resources when comparing
to original "Burden of Proof"-process figure \ref{fig:burdofpr} on page
\pageref{fig:burdofpr}, even lot of needed elderly people relatives
helping effort is not drawn visible to process
figure \ref{fig:burdofpr} \cite{VMAuroraAiNakokulmiaIhmiskeskeisyyteen}.
Result is that peoples give up and do not get available financial support
\cite{ElakelaisetRynJanKoskimiesJaElakeliitonIreneVuorisalo}\-\cite{ElakelaisetEivatTieda}\-\cite{StudentHousingAllowance}\-\cite{Many_Stepwise_Rules_Leads_To_Mess}...
system just increase bureaugratic overhead.
System hast to fixed or removed!

\begin{figure}
 \begin{center}
  \epsfig{figure=figures/digitali.eps,width=\textwidth}
  \caption{Digitalized process example}
  \label{fig:digitalized} \index{process digitalized}
 \end{center}
\end{figure}

\subsubsection{Passed people genealogy example}
To sort out passed people things you has to have DVV report from family relationships,
and you have to request DVV's fully digitally generated report from DVV by yourself during mourning,
and lot of other changes due perhaps the 30\euro~price.
Even governemnt will tax passed people assets transfer further anyway,
and they mostly know due police and/or doctors created passing out infos
and automatic generation should start immediately on DVV.
Possible other registers, from curch books, could be initiated as well in most cases.
Costs should be taken from inheritance tax and do this step automatically,
transfer money later though budget what inheritance tax feeds.
I hope that this improvement\cite{SukuselvitysPerukirja} will proceed.

\subsubsection{Examples summary and some other findings}
\label{examples_summary_and_findings}
\index{examples summary and findings}
I am quite sure that if we check more these "services" we find more and more
direct digitization of old process without real digitalization.
Every service I lately have personally touched has had these problems with the
procesesses. This is national shame, at least should
be \cite{Many_Stepwise_Rules_Leads_To_Mess}\-\cite{StudentHousingAllowance}\-\cite{ElakelaisetEivatTieda}...
And it has to be fixed, either by digitalizing processes from customer
perspective or by merging and removing unnecessary rules, payments.

\paragraph{Social Insurance Institution (SII) card}
\label{SII_card}\index{SII card}
Releated to social security insurance we get repeative request for the persons
Social Insurnce Institution (SII) card ID "SOTU", even if you are used strong
identification delivering personal identity code "HETU" to log into system or
shown identity card for personnell using fully to state services integrated
systems. It's frustrating finding from how base level issues are still not
solved on our "digitalized" service paths.

It tells that there is no interface nor integration or legal support to check;
does person personal identy number "HETU" have related social security
insurance identity "SOTU" at social insurance institution (SII). SII "KELA" is
the same organization delivering native insurences and the base "KANTA" against
most most systems are integrated to deliver information to e-perscriptions or
epicrisis. Why don't deliver social insurance existense status as well?

There is still some improvements needed between DVV and KELA application
interfaces and integrations. For native Finnish citizen social security
insurance identity number "SOTU" is same than native personal identity number
"HETU". Seem to be needless to ask does those services work with an Electronic
Unique Identification Number "SATU", which would be better, more secure for
the electric transactions, because it can be replaced on case of identity teft.

\paragraph{Standardized clinical operation description, metadata}
\label{clinical_data}\index{clinical data}

We have good start around KANTA-base, some existing working processes, some
verbal process descriptions in practice. What is lacking is target process
descriptions and descriptions from the next steps, step by step through the
service paths, use case by use case\cite{LifeEventDetailDrivenProcessQuides}.
Localized common clinical data\cite{HealthEventsActionsMetaData}; action,
operation event common standardized descriptions are totally missing, from
citizen perspective, it seems that there isn't much reasonable cooperation
happening in the Finnish scene related to this. At least Finnish and Swedish
language translations from european or global dataset are not visible
publicitly as should be, if we want to have common standardized dataset used
to define patient service paths, dividing actions and operations to most
relevant place of welfare services area. This also improve possibility to use
any other EU country services, or service from abroad, if they are cheaper,
than locally available ones. Common metadata is allowing public and
commercial service providers available resources and timeslots, most
technososioeconomic way optimized resource allocation and use.

\paragraph{Common course description data through all levels}
\label{course_data}\index{course data}

Similarly we are lacking common structured course content, metadata and
courses structural data for our education system
\cite{CoursesAndRequirements} in country which population could fit to single
city and which new generations are from year to year smaller than the previous
ones.

\paragraph{Summary from education cooperation gaps}
\label{education_cooperation_gaps}\index{education cooperation gaps}

Without the cooperation and standardization on healtcare and education leads
to situation where same, or at least overlapping work, is done on several paid
projects. As nation do we have enough wealth to pay from repeative, partly not
needed, overlapping, frankly said; semi\-productive tasks slowing the
development down? Private parties do not complain because repeatition keeps
they order book full. From year to year smaller generations and broader study
variations need clearly visible trend which will force units to close if
reasonable cooperation and specialization is not done early enough.

In future it's more important that peoples can choose, cherry pick, part of
courses to make things they are keen to develop. Peoples learning capasity
is not grown so much that they can blindly learn everything, therefore
creation of new success stories need wide education material and courses
supply with possibility to cherry pick courses for they own ideas, dreams
to come true, for real success stories at megatrends, eco-trends, space
exploration and utilization.

We need reasoable size learning of old and then something new need to
be combined. Whole package should be one person imaginable, understable.

We need cooperation and standardization on learning courses metadata and
content to manage more with less and be productive doing new thigs with
the saving benefits we got from cooperation.

\subsection{Uneven load and reimbursement on public and private tracks of our healt care}
\label{unbalance_on_two_tracks}
\index{unbalance on two tracks}
There has been at least four decades continued chaining and centralization on
private healthcare services side. It has brought good things, but same time
it's brought also bad things because all employer financed employee helthcare
services are well financed and mainly care basically health peoples. Peoples
having serious health issues usually get sickness pension and drop over the
health area health services. So health areas have more problematic customers
who are scattered around large regional areas. Some communs have very high
average age on areas in the middle of nowhere. Column from Heikki Hiilamo
\cite{HiilamonKolumni202411250645} points clearly this during the years
collected unbalance.

Solution is easy; All health services, including employer paid, are set to
same line. Law is changed so that employer payment is take as part of company
taxing and health areas having university hospitals are then arranging the
services, by themself, with the other health areas, and by purchasing from
existing competitive pricing capable service providers. Heikki Hiilamo
pointed out bit more polaid taxing change\cite{YleAtalk202412122106} where
employer health services are only changed to be taxed benefits, which makes
public and private services more comparable related to financing and customers
average heath situation on both user groups.

But even in this case
there is no need for 25 separate health service areas, here including Finnish
Student Health Services (FSHS/YTHS) and Åland Islands as additional healt
areas besides the existing 23, which is much more than we have university
hospitals (Helsinki, Tampere, Turku, Kuopio, Oulu) already having management
structures, understanding from education resources and possibilities for
new medical professionals training. Therefore national wide resourcing
decisions are left for the university hospitals and Ministry of Social
Affairs and Health. This should include evaluation for the overall need for
these 25 health service areas. All 25 areas in this country which population
equals to one big city!

It should be enough to do service optimization so that for service total cost
includes customer resources use. In practice meaning that each hour customer
has to use to get service is priced at least with the minimum wage or actual
wage if being in work, and including all travel time, travel costs from door
to door and back. With this kind of finacial evaluation we maintain reasonable
near services, centralize regional and national service socially and
economically viable manner. By giving in middle of nowhere living person used
time value by counting it as worktime we take human value in account
\cite{Yle202411251731}.

In practice meaning that basic healtcare services are brought near by even
there is only few customers, because calculation points it feasible when
everyone's resource consumption is taken into account. Return of "own medical
doctor model" is already
proposed\cite{Yle202410222121}\cite{YleAstudio202410292107} and
supported\cite{Yle202411191722} to as solution for basic problems with the
existing system. Own doctor model reduces the effects of poor integrations and
processes in system, because same professional person use own memory and own
local system which is there still between the visits, no matter does the data
been delivered to kanta-base or not. Still data must be delivered to
kanta-base for case of move or other need to get service at somewhere else
where also access to persons stored health data, epicrisis is needed.

\subsection{Target situation}
\label{target_situation}
\index{target situation}

All processes are gone through overall technososioeconomical optimization,
fully rewised, rewritten, simplified processes, look\-ed from both producer
and customer perspectives when cheking overall technosocioeconomical
optimization from national, european union, and in some cases from global
perspective.

Thinking through the service paths, individuals organizations and processes
along citizens service path. Special interest on standardization, removal
of overlapping efforts and actual information flow, how it usually goes now;
\hfill\break\begin{center}
digital(text)~-~paper(text)~-~digized(pixels)~-~digital(text)
\cite{VM006_00_2024},\end{center}
and how we can support citizen to do most with less, meaning that data goes
from service provider to Kanta, or if it goes through citizens message services
then receiving end should get it as text-data not as picture-pixels someone
has to digitize by hand as work time process. If not any better methods found
for filled forms then pure ASCII/ISO 8859-1 text-form would be enough good for
the process because it can be filled and forwarded by citizen as well as
healtcare worker etc., and it can be signed and forwarded electronically.
Anyhow Kanta-base should be preferred and citizen could get message or paper
copy to remember it.

\subsection{Expectations}
\label{expectations}
\index{expectations}

We have seen govenment enterprice architecture exercise, human centric
artificial intelligence program and Finninsh municipalities association's good
work around government and municipalities enterprice arhitecture. It's good
to see human centric open govenmental digital architetecture activities.
Public money -- public code, public data -- public service, public society --
public architecture are good values to maintain cooperation and common wealth,
because everybody's contribution can multiply common output and this way bring
civil society efforts to common use, same time reducing overall cost and time
what takes to get results and control position for democratically selected
government from society's own digital service environment.

The fear is that human centric enterprice arhitecture development is done in
project mode leaving results unmaintained. Enterprice arhitecture management
and development is CRITICAL SERVICE, not a project. Architecture all time
evolving at least now when we live era of fast digitalization. We need current
state, target state, gap and next development step visibility online to be up
to date, and discussion from next steps ongoing for sub areas needing
development work. It's really hard task to get needed human centric over the
organizational limits happening process digitalization thinking going into
practice in our silo-organizations sitting peoples heads. It requires from
managment that they are capable to set up virtual teams over organizational
limits on need bases. This has to be understood and approved on upper
management levels, and we really need to get it going into execution on our
practical day to day work.

Worst case is that we have not done our homework and some profitmaker makes
promise to our politicians who do not understand, and during investement
implementation vendor locking is created, valuable to whole invested money
and more nearly as good as natural monopoly, and then this is monetized to
profitmakers benefit and to our society's loss in form of oversized operation
and further development investment expenses. And it's not first time our
publically superviced natural monopoles are sold out for money making, so
need to be aware.
%-END OF INCLUDE FILE----------------------------------------------------------
