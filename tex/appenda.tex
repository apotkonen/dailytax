%------------------------------------------------------------------------------
%
%	appendix.tex Available income statistics usage
%
%	INCLUDE FILE FOR LaTeX2e DOCUMENT
%
%	AUTHOR: Ari Potkonen /JARVENPAA/ Mon Jun 28 2022
%------------------------------------------------------------------------------
%         1         2         3         4         5         6         7
%123456789012345678901234567890123456789012345678901234567890123456789012345678
%-BEGIN OF INCLUDE FILE--------------------------------------------------------

\chapter{Income statistics}
%\addcontentsline{toc}{chapter}{Income statistics}
\index{statistics}
\label{statistics}
In Finland some statistics are available at \url{http://stat.fi}.
You can find income statistics from \url{http://stat.fi/tilasto/tjt}.
There is a table "11wh" -- "Income shares (\%),
means, medians and maximum values of decile and percentile groups, 1995-2021".
After editing your query, you can save it and actual query result.
It's also possible to edit and repeat query from commandline in simplistic form
using "curl -d @query.json" or "wget --post-file=query.json" from commandline.

From those available statistics we can get consumption units
$u$ grou\-ped to fractile groups $g$,
and when we sum up these fractiles we get number of consumption
units\cite{ConsumptionUnitD}
in population, which should be about the number of citizens in country.  

\begin{equation} \label{eq:fractile_units_sum}
	\sum_{0\%}^{100\%} u =
	\sum_{0\%}^{5\%} u +
	\sum_{5\%}^{10\%} u +
	...+
	\sum_{95\%}^{100\%} u = U \approx citizens
\end{equation}

Besides the percentage limits of fractile group $g$
these limits are also available on income $i_{u\%}$ -values from edge of each fractile,
so we can define people group $g(i)$ size on each group edge where
$i=i_{u\%}$ and median values also available from stats,
and last top notch which is removed from statistics is available from gossib tabloids
$i=i_{top}$.

\begin{equation} \label{eq:fractile_units_group}
	g(i_{u0\%},i_{u\%})=
        \sum_{i=i_{u0\%}}^{i_{u\%}} u =
	\sum_{i_{u0\%}}^{i_{u5\%}} u+
	...+
	\sum_{i_{u\%-5\%}}^{i_{u\%}} u
\end{equation}

\begin{equation} \label{eq:units_group}
	g(a,b)=\sum_{i=a}^{b} u
\end{equation}

\begin{equation} \label{eq:group}
	g(i)=g(0,i)
\end{equation}

\begin{equation} \label{eq:units_group_max}
	g(0,i_{top}) =
%	g_{max}(i) = 
	U \approx citizens
\end{equation}

\begin{equation} \label{eq:units_group_propability}
	P(a,b)=\frac{g(a,b)}{g(0,i_{top})}
\end{equation}

\begin{equation} \label{eq:group_propability}
	P(i)=P(0,i)
\end{equation}

\begin{equation} \label{eq:overall_propability}
	P(i_{top})=1
\end{equation}

\begin{equation} \label{eq:delta_g_delta_i}
	\frac {g(i_{u+2.5\%})-g(i_{u\%})}{i_{u+2.5\%}-i_{u\%}}
	=\frac{\Delta g}{\Delta i}
\end{equation}

\begin{equation} \label{eq:group_density}
	\frac{\Delta g}{\Delta i}~
	=_{\Delta i \rightarrow 0}
	~ g'(i)
\end{equation}

\begin{equation} \label{eq:propability_density}
	p(\frac{i}{i_{top}})=\frac{g'(i)}{g(0,i_{top})}
\end{equation}

\begin{equation} \label{eq:propability_distribution}
	P(i_{top})=\sum_{i=0}^{i_{top}} p(i)=1
\end{equation}

\begin{equation} \label{eq:propability}
	P(i)=\int_{i=0}^i p(i)
\end{equation}

\begin{equation} \label{eq:propability_overall}
	P(\infty)= P(i_{top})= 1
\end{equation}

\begin{equation} \label{eq:propability_overall_infty}
	P(0,\infty)=\int_{i=0}^{\infty} p(i)=1
\end{equation}

%\begin{equation} \label{eq:GrossIncome}
%	G=\int_{i=0}^{i_{top}} i g'(i) =
%	g(0,i_{top}) \int_{i=0}^{i_{top}} i p(i)
%\end{equation}

From statistics is possible to get some information to try find suitable
PDF (Propability Distribution Function) $f(i)$ to be fitted into collected $g'(i)$
points and respective CDF (Cumulative Density Function) F(i) to fit into $g(i)$ points.
Suitable function candidates can be found based to experience or by using Python-Fitter
and/or -Distfit-software to do Sum of Squared Errors (SSE) or Residual Sum of Squa\-res (RSS).
Log-logistic distribution\cite{LogLogisticDistribution} (Peter R. Fisk\cite{Fisk})
would be good starting point to look suitable distribution function.

%\begin{equation} \label{eq:Alpha} \alpha=\tilde{g'} \end{equation}
\begin{equation} \label{eq:PDFfisk}
 PDF_{fisk}(i,\alpha,\beta)=\frac{(\beta/\alpha)(i/\alpha)^{(\beta-1)}}{(1+(i/\alpha)^\beta)^2}
\end{equation}
\begin{equation} \label{eq:PDFf}
 f(i;i_0,\alpha,\beta)=
 \alpha\frac{(\beta/\alpha)((i-i_0)/\alpha)^{(\beta-1)}}{(1+((i-i_0)/\alpha)^\beta)^2}
\end{equation}

%http://wikipedia.org/wiki/Log-logistic_distribution#Characterization
\begin{equation} \label{eq:CDFfisk}
 CDF_{fisk}(i;\alpha,\beta)
 =\frac{1}{1+(i/\alpha)^{-\beta}}
%=\frac{(i/\alpha)^\beta}{(i/\alpha)^\beta+1}
%=\frac{i^\beta}{i^\beta+\alpha^\beta}
%=F(i;\alpha,\beta)
\end{equation}

\begin{equation} \label{eq:CDFF}
 F(i;i_0,\alpha,\beta,U)
 =\frac{U}{1+((i-i_0)/\alpha)^{-\beta}}
%=\frac{U((i-i_0)/\alpha)^\beta}{((i-i_0)/\alpha)^\beta+1}
%=\frac{U((i-i_0))^\beta}{((i-i_0))^\beta+a^\beta}
%=\frac{U(i-i_0)^\beta}{(i-i_0)^\beta+a^\beta}
\end{equation}

To be able to fill database for testing with the data, what known propability
distribution represents, we could use inverse transform sampling from interval
$(0,1]$. Given random numbers are interpreted as propability and mapped with
inverse cumulative distribution to data points\cite{InvTransformSampling}.
In this case it's also possible to sample consumption units group $g \in [1,U]$
size from one unit-person to number of citizens, and then do inverse transform
with scaled cumulative distribution function CDF inversion $F^{-1}(g)$ .

%http://wikipedia.org/wiki/Inverse_transform_sampling
\begin{equation} \label{eq:Propability}
 P(g;U)=\frac{g}{U} \in [1/U,1] ~;~ g \leq U ~;~ g,U \in \mathbb{N}^+
\end{equation}

\begin{equation} \label{eq:CDFfiskInverse}
%CDF^{-1}_{fisk}(P;\alpha,\beta)=i(P;\alpha,\beta)=\alpha\left(\frac{1-P}{P}\right)^\beta
 CDF^{-1}_{fisk}(P;\alpha,\beta)=i(P;\alpha,\beta)=\alpha\left(\frac{1}{P}-1\right)^\beta
\end{equation}

%http://wikipedia.org/wiki/Inverse_transform_sampling
\begin{equation} \label{eq:FInverse}
 F^{-1}(g;\alpha,\beta,U,i_0)
%=i(g;\alpha,\beta,U,i_0)
 =\alpha\left(\frac{U}{g}-1\right)^\beta+i_0
\end{equation}

%-END OF INCLUDE FILE----------------------------------------------------------
